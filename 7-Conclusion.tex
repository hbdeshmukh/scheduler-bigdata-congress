% !TEX root = scheduler.tex
\section{Conclusions and Future Work}\label{sec:conclusion}
%Query scheduling is an important problem and its importance is growing with the emerging trends in cloud databases.
In this paper we present a scheduler framework that uses a design based on separation of policy and mechanism to produce a scheduler that can support a wide-range of policies, even in dynamic workload settings and without requiring complex and accurate estimates from a query optimizer. The proposed scheduler framework is holistic as it also incorporates a load control mechanism. We have implemented our methods in an open-source in-memory database \sys{}, and also demonstrated the effectiveness of our approach.


%interesting properties such as resource allocation using fair and priority-based policies and in-built load control mechanisms. We demonstrate the effectiveness of the scheduler in enforcing the above policies with the SSB workload with uniform and skewed dataset. Our fine-grained task scheduling paradigm allows the scheduler to be reactive to unexpected situations such as arrival of a new high-priority query or sudden burst in resource demands.

There are a number of interesting directions for future work, including extending the scheduler framework to the distributed version of \sys{}. %, and allowing more kinds of resources such as disk I/O and network and exploring more estimation techniques for our Learning Agent. 