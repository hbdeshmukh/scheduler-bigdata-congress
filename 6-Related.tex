% !TEX root = scheduler.tex
\section{Related Work}\label{sec:related}
We now describe the related work on scheduling in database systems, operating systems (OS), and networks. %, and cluster computing. 

Our \textit{work orders} abstraction is similar to other abstractions like \textit{morsels} in Hyper~\cite{morsel} and the \textit{segment-based parallelism}~\cite{wang2016elastic}. 
All these abstractions propose a mechanism to achieve high intra-query, intra-operator data parallelism.
Hyper~\cite{morsel} uses a \textit{pull-based} scheduling approach i.e. workers \textit{pull} work (morsels) from a pool. 
%Every worker thread continues to pull a morsel from a global pool, and executes it.
We use a \textit{push-based} model, where the scheduler controls the assignment of work to workers.
The pull-based dispatch model suffices for executing one query at a time. 
However, a push-based model can be simpler to implement sophisticated functionalities such as priority-based query scheduling, incorporating a flow control across multiple pipelines, (as implemented by Wang et al.~\cite{wang2016elastic}), cache-aware or locality-aware task assignment.

The elastic pipelining implementation~\cite{wang2016elastic} uses a scalability vector to vary the degree of parallelism of segments of the query plan.
The scalability vector tracks the query performance when number of cores are varied,  and it does not use any prediction technique.
Their objective is to maximize the performance of a single query executed on a cluster.
Our policies are implemented for sharing resources among concurrent queries.
We enforce such policies using a learning-based approach, that can accommodate estimates provided by other techniques.

There is related work~\cite{gupta2009fair} on ordering queries in a workload with different objectives such as fairness, effectiveness, efficiency and QoS. 
This work is complimentary to our scheduler design as it deals with ordering the queries \textit{before} they enter the system, where as we focus on scheduling \textit{admitted} queries. 
%\textit{Qshuffler}~\cite{ahmad2011interaction} is a query 
%scheduler for report generation workloads. 
%It clusters queries based on their interaction with each other and predicts the 
%performance of the mix using statistical methods.
%It does not preempt queries once they are scheduled, whereas \sys{}'s 
%scheduler uses fine grained control over query execution, allowing query suspension and resumption. 

%In business intelligence settings, workload management is an important challenge.
%Some techniques to manage workloads are based on QoS considerations. 
%Krompass et al.~\cite{krompass2007dynamic} identify and handle mis-behaving queries in a workload, and also propose an economic model for prioritizing and penalizing queries based on their service level agreements~\cite{krompass2006quality}.
%Such techniques can be combined with our scheduler. 
%For instance, our load-controller's action on a query that uses large amount of memory is to suspend it. 
%Alternate actions could be to re-prioritize, kill, or resubmit the query, as described in~\cite{krompass2007dynamic}.
%\sys{}'s prioritized query execution approach can be effective in  scenarios involving 
%SLAs. 

Workload management is an important problem for commercial databases~\cite{res_gov, rm, DB2, teradatawm, gpdb, hpwm}. 
A common theme in commercial workload management solutions is to provide admission control capabilities, classification of queries based on their estimated resource requirements (e.g. CPU, memory, I/O), encoding resource allocation limits as resource pools and mapping workloads to the resource pools. 
While such estimation methods can be used to complement our approach, 
our scheduler can also work without such detailed estimation techniques. 
Prior research in this area~\cite{krompass2007dynamic, krompass2006quality} has focused on identifying misbehaving queries, prioritizing/penalizing queries to meet the service level objectives.
Our load controller implementation can be complemented with such functionalities.

%Many commercial databases systems offer workload management solutions~\cite{res_gov, rm, DB2, teradatawm, gpdb, hpwm}.

Resource management, in particular memory management is a crucial problem for database systems. 
Prior work in this area~\cite{mehta1993dynamic, davison1995dynamic} has focused on dynamic memory allocation schemes.
%Mehta and Dewitt's~\cite{mehta1993dynamic} memory allocation scheme
%grouped queries by their estimated memory requirements, and allocated memory to different query classes. 
%Davison and Graefe introduced a resource brokering model~\cite{davison1995dynamic} to minimize query execution times with a constraint of fairness.
These techniques are complimentary to our execution engine and can also be incorporated in our buffer manager. 

Predicting query execution time is an active area of research. 
Wu et al.~\cite{wu2013towards, wu2014uncertainty} developed an analytical model using optimizer's cost model to estimate the CPU and I/O costs for individual query, and used a queueing model to estimate the execution times of concurrent queries in a workload.
Duggan et al.~\cite{duggan2011performance} presented a model to estimate the performance impact of running concurrent queries.
Our Learning Agent's prediction accuracy can benefit from such models, but can also function without them. 
%Despite these advances in the area of execution time estimation, using 
%inaccurate estimates for achieving fairness may result into unfair schedules. 

%There is a significant interest in minimizing workload execution time by sharing 
%data and computations across queries.
%QPipe~\cite{harizopoulos2005qpipe} exploits common data and operations among 
%different queries. 
%CJoin~\cite{candea2009scalable} operator introduced by Candea et al. continuously 
%scans the fact table, applies predicates from different queries and routes the 
%results to individual queries.
%Zukowoski et al. proposed \textit{co-operative scans}~\cite{zukowski2007cooperative} 
%to share scan operations among concurrent queries. 
%%The sharing of scans reduces the number of I/O operations, increases the data 
%%sharing among concurrent queries thereby improving the latency of individual 
%%queries and the overall execution time. 
%SharedDB~\cite{giannikis2012shareddb} creates a single global query plan for a 
%workload to share computations across queries, thus making it suitable for high 
%throughput demanding environments. 
%These sharing approaches treat every query equally and it is not clear if they 
%can respect query priorities.
%Work and data sharing in \sys{} can be achieved by changing the way work orders are
%created, e.g. if multiple queries need to scan the same block, a single work order can
%be created for all of them. 
%In practice, we create separate work orders for different queries to keep the 
%execution simple.
%Additionally, many database systems \reminder{Can we use examples of Postgres, 
%Greenplum, SQL server, Oracle here?} still follow the \textit{query-at-a-time} 
%model.
%In such systems, our scheduler paves a way towards policy-driven execution
%of concurrent queries. 

Scheduling problem has also been studied in the OS and the networks community. 
%Kay and Lauder~\cite{kay1988fair} described \textit{Share}, a pioneering scheduler that ensured fairness for all the system users.
Our probabilistic framework for scheduling is inspired by the seminal lottery scheduling~\cite{lottery-scheduling} % by Waldspurger and Weihl,
in which different processes are assigned certain number of lottery tickets,
A lottery is conducted after every fixed time intervals and the process that wins the lottery gets to execute in the next time slice. 

A key difference in lottery scheduling and our work is that the OS scheduling is usually preemptive. %, meaning that a process can be preempted after it uses its time slice. 
The OS maintains a process context that describes the progress of the preempted process along with the relevant metadata. 
\sys{}'s scheduling is non-preemptive, which means once a work order begins its execution on a CPU core, it continues to do so until completion.
Non-preemptive scheduling provides us an exemption for maintaining work order context (similar to process context).
This design choice helps simplifying our core relational operator execution algorithms.

%\reminder{Removed OS refs}
%Meehan~\cite{meehean2011towards} analyzed various Linux CPU schedulers, %like O(1), CFS, and BFS. 
%showcased the issues of black box CPU scheduling and highlighted the need for increased transparency in CPU scheduling.
%%It achieves fairness by assigning equal proportion of time to the users and not 
%%to the individual processes that are running on the system. 
%%Pabla et al.~\cite{pabla2009completely} described how the Completely Fair Scheduler (CFS) strives to be fair to all the processes running in the Linux. 
%Peter et al.~\cite{peter2010design} proposed design principles for multi-core schedulers used in general purpose OS.
%Giceva et al.~\cite{giceva2013cod} advocated co-designing database and OS for their better integration.

\sys{}'s scheduler design aligns to the idea that scheduling must be transparent, fine-grained, and most importantly the need to separate mechanisms from policies~\cite{LampsonS76} - a common theme found in the OS literature.

Deficit Round Robin (DRR)~\cite{shreedhar1996efficient} is a technique for network packet scheduling.
Our usage of work order execution time as a metric is similar DRR's usage of packet sizes. 
However DRR scheduling is inherently round robin based (with additional maintenance of quantum information), where as our scheduling is based on dynamic probabilities. 

%\reminder{Removed cluster scheduling refs}
%Cluster scheduling has received a wide attention recently.
%Isard et al.~\cite{isard2009quincy} encoded the problem of scheduling jobs on compute nodes in the cluster as a graph that captures the data-locality and fairness requirements of the jobs. 
%\textit{Tetris}~\cite{grandl2014multi} is a multi-resource cluster scheduler that packs tasks to machines based on their demands for different resources. 
%This paper focuses on a single node setting, however, such ideas may be interesting to purse in the distributed version of \sys{}, which is part of future work.
%In cluster scheduling, resource demands of tasks are usually well understood. 
%Therefore variations in the duration of task executions can be statically modelled 
%- e.g. changing the site of task execution, time for input data movement across 
%network, etc. 
%In contrast, predicting cardinality and time estimates in database systems is 
%still an active research area, therefore accurately modeling the entire query 
%execution before scheduling the query may be difficult.
